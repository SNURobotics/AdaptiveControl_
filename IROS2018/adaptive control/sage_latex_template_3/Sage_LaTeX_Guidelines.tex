% sage_latex_guidelines.tex V1.10, 24 June 2016

\documentclass[Afour,sageh,times]{sagej}

\usepackage{moreverb,url}

\usepackage[colorlinks,bookmarksopen,bookmarksnumbered,citecolor=red,urlcolor=red]{hyperref}
\newtheorem{proposition}{Proposition}
\newtheorem{remark}{Remark}


\newcommand\BibTeX{{\rmfamily B\kern-.05em \textsc{i\kern-.025em b}\kern-.08em
T\kern-.1667em\lower.7ex\hbox{E}\kern-.125emX}}

\def\volumeyear{2016}

\begin{document}

\runninghead{Lee and Park}

\title{Smooth, Physically-Consistent Adaptive Control for Robot Manipulators}

\author{Taeyoon Lee\affilnum{1} and Frank C. Park\affilnum{2}}

\affiliation{\affilnum{1}Sunrise Setting Ltd, UK\\
\affilnum{2}SAGE Publications Ltd, UK}

\corrauth{Alistair Smith, Sunrise Setting Ltd
Brixham Laboratory,
Freshwater Quarry,
Brixham, Devon,
TQ5~8BA, UK.}

\email{alistair.smith@sunrise-setting.co.uk}

\begin{abstract}
This paper describes the use of the \LaTeXe\
\textsf{\journalclass} class file for setting papers to be
submitted to a \textit{SAGE Publications} journal.
The template can be downloaded \href{http://www.uk.sagepub.com/repository/binaries/SAGE LaTeX template.zip}{here}.
\end{abstract}

\keywords{Class file, \LaTeXe, \textit{SAGE Publications}}

\maketitle
\section{Introduction}
\newpage
\section{Geometry of Rigid Body Inertial Parameter}
\subsection{Riemmanian Geometry of Rigid Body Inertial Parameter}
Inertial parameters that represent the dynamics of a single rigid body consist of mass $m$, position of center of mass $p_{b}\in\mathbb{R}^{3}$ and rotational inertia matrix $I_{b}\in\mathcal{S}(3)$, all represented in some body frame $\{b\}$. For the purpose of exploiting the linearity of inertial parameters in dynamic equations for various dynamics-based applications including dynamic identification and adaptive control, it is common to express them in the vectorized form,
\begin{equation*}
\phi_{b} = [m,h_{b},I_{b}^{xx},I_{b}^{yy},I_{b}^{zz},I_{b}^{xy},I_{b}^{yz},I_{b}^{zx}]\in\mathbb{R}^{10}
\end{equation*}
, where $h_{b} = mp_{b}\in \mathbb{R}^{3}$. It is clear that the dimension of inertial parameters of a single rigid body is ten. However, they are not valid over the whole $\mathbb{R}^{10}$ space but only the subset, since it should satisfy so called physical consistency condition, meaning that at least a single nonnegative mass density function $\rho : \mathbb{R}^{3} \rightarrow \mathbb{R}^{+}_{0}$ should exist that realizes the given the parameter $\phi_{b}$.

As pointed out in \cite{c1}, the requirement that a rigid body's mass $m$ be positive
and its inertia matrix at the center of mass, $I_{b}^{C}=I_{b}-m[p_{b}][p_{b}]^{T}$, be positive-definite is in fact a necessary, but not sufficient. So called triangular inequality condition on the three eigenvalues of $I_{b}^{C}$ has to hold additionally in order for the nonnegative mass density function to exist. 

In \cite{c2} the same physical consistency condition is expressed in the following
alternative but equivalent form: the $4 \times 4$ symmetric matrix $P_{b} \in \mathcal{S}(4)$ 
is defined as 
\begin{eqnarray}
P_{b} = \int \left[\begin{array}{c} \vec{r}_{b} \\ 1 \end{array}\right]
  \left[\begin{array}{c} \vec{r}_{b} \\ 1 \end{array}\right]^{T} \rho(\vec{r}_b) dV_{b} 
= \left[\begin{array}{cc} \Sigma_{b} & h_{b} \\ h_{b}^{T} & m \end{array}\right],
\label{44psd}
\end{eqnarray}
where the second moment matrix $\Sigma_{b}\in \mathcal{S}(3)$ is defined by
$\Sigma_{b} = \int \vec{r}_{b}\vec{r}_{b}^{T} \rho(\vec{r}_{b})dV_{b}$.  It is shown
in \cite{c2} that the condition that $P_b$ be positive-definite, i.e.,
\begin{equation}
P_b \succ 0
\label{full_cond_44psd}
\end{equation}
is equivalent to the physical consistency conditions described in \cite{c1}.
To make this identification more explicit, from the relation $\Sigma_{b}
= \frac{1}{2}\mathrm{tr}(I_{b})\mathbf{1} - I_{b}$, which holds for any choice of
body frame $\{b\}$, the one-to-one linear mapping $f : \mathbb{R}^{10}
\rightarrow \mathcal{S}(4)$ can be defined as follows:
\begin{align*}
&f(\phi_{b}) = P_{b} = \left[\begin{array}{cc} \frac{1}{2}\mathrm{tr}(I_{b})
\cdot\mathbf{1} - I_{b} & h_{b} \\ h_{b}^{T} & m\end{array}\right] \in \mathcal{S}(4)\\
&f^{-1}(P_{b}) = \phi_{b}(m, h_{b}, \mathrm{tr}(\Sigma_{b})\cdot\mathbf{1}
- \Sigma_{b}) \in \mathbb{R}^{10}.
\end{align*}
The physical consistency conditions on the  parameter $\phi_{b} \in \mathbb{R}^{10}$
can be identified under the mapping $f$ with the requirement that the symmetric matrix
$P_{b} = f(\phi_b) \in \mathcal{P}(4)$ be positive definite.

Based on the above, we define the manifold $\mathcal{M}$ of the set of physically consistent
inertial parameters for a single rigid body as follows:
\begin{align*}
\mathcal{M} &\simeq \{ \phi_{b} \in \mathbb{R}^{10} : f(\phi_{b}) \succ 0\}
\subset\mathbb{R}^{10}\nonumber\\
& \simeq \{ P_{b} \in \mathcal{S}(4) : P_{b} \succ 0\} = \mathcal{P}(4).
\end{align*}
The elements can be identified in both $\mathbb{R}^{10}$ and $\mathcal{P}(4)$, also
for different choices of body-fixed reference frame $\{b\}$. For a multibody system
with $n$ rigid links, the space of physically consistent inertial parameters is
given by the product space $\mathcal{M}^{n} \simeq \mathcal{P}(4)^{n}$.

We can explore the Riemmanian geometry of manifold $\mathcal{M}$ by defining a natural choice of Riemmanian metric on $\mathcal{M}$ as in []. In order for a distance to be naturally defined on a manifold, it should be invariant with respect to choice of coordinate frames or physical units/scale. It would also be desirable if the distance were to possess a physical meaning that corresponds to our intuition. The standard Euclidean metric on $\mathcal{M}$ under the vectorized representation $\phi_{b}$ satisfies none of these desiderata. As was proven in [], geodesic distance metric that derives from the affine-invariant riemannian metric on positive definite matrix provides a coordinate invaraint metric on $\mathcal{M}$ under the representation $P_{b}= f(\phi_{b})\in\mathcal{P}(4)$ as,
\begin{align}
d_{\mathcal{M}}({^{1}}\phi_{b}, {^{2}}\phi_{b})
&= d_{\mathcal{P}(4)}({^{1}}P_{b}, {^{2}}P_{b})
\label{metric_singlebody}\\
&= \bigg(\sum_{i=1}^{4}\big(\log(\lambda_{i})\big)^2\bigg)^{1/2}
\end{align}
, where $\lambda_i$ are the eigenvalues of ${^{1}}P_{b}^{-1/2}\cdot{^{2}}P_{b}\cdot{^{1}}P_{b}^{-1/2}$ or equivalently those of ${^{1}}P_{b}^{-1}\cdot{^{2}}P_{b}$. We now omit all the coordinate frame subscripts for expressing the inertial paramters $\phi$ and $P$. Additional physical aspects of the distance metric, that relates to Fisher information metric on the space of Gaussian density functions, is discussed in []. 
\subsection{Bregman Divergence as Pseudo Distance Metric}
In many computational applications, a Riemmanian geometric approach provides a mathematically strict set of tools to treat data that is sensitive to the coordinate choices. However, it can sometimes result in non-tractable computations or not fit well into the specific applications. As will be specified in the following sections, Riemmanian geodesic distance defined as (\ref{metric_singlebody}) fails to be directly applicable to the existing framework of robot adaptive control.

In this section, we propose an alternative psuedo distance metric on inertial parameters that derives from the Bregman divergence of a log-det function on a positive definite matrix, which still preserves coordinate-invariance property but is only non-negative and fails to be symmetric nor satisfy triangular inequality.

Bregman divergence or distance associated with function $F: \Omega\rightarrow \mathbb{R}$ for points $p,q\in\Omega$ is the difference between the value of $F$ at point $p$ and the value of the first-order Taylor expansion of $F$ around point $q$ evaluated at point $p$:
\begin{equation*}
D_{F(\Omega)}(p || q) = F(p)-F(q)-<\nabla F(q), p-q>.
\end{equation*}
When $\Omega = \mathcal{P}(4)$, the bregman divergence associated with a log-det function $F(P) = -\log|P|$ for $P\in\mathcal{P}(4)$ is given by,
\begin{align}
D_{F(\mathcal{P}(4))}(P || Q) &= \log\frac{|Q|}{|P|} + \mathrm{tr}(Q^{-1}P) - 4\\
&= \sum_{i=1}^4(-\log(\lambda_{i})+\lambda_{i}-1)
\end{align}
, where $\lambda_i$ are the eigenvalues of $Q^{-1}P$ or equivalently $Q^{-1/2}PQ^{-1/2}$. Note that $D_{F(\mathcal{P}(4))}$ is affine-invariant, that is invariant under the $GL(n)$ group action, i.e., $G * P = GPG^{T}$ for $G\in GL(n)$ and $P\in\mathcal{P}(4)$.

Again using the embedding $f$ from $\phi$ to $P = f(\phi)$ we can define the distance measure on $\mathcal{M}$ as
\begin{equation}
D_{\mathcal{M}}({^{1}}\phi_{b}, {^{2}}\phi_{b}) = D_{F(\mathcal{P}(4))}({^{1}}P_{b} ||  {^{2}}P_{b}) \label{Bregman_div}
\end{equation}
, which is coordinate-invaraint, since the coordinate transformation on $P$ follow the exact form of $GL(n)$ group action as described in [].
(((((Relation to riemmanian metric and KL-divergence on Gaussian distributions)))))


\section{Previous Results on Adaptive Control for Robot Manipulators}
Here we revisit well-known globally convergent adaptive control methods developed for robot manipulators. Given the dynamic equation of general robot manipulators of the form,
\begin{equation}
M(q,\Phi)\ddot{q}+C(q,\dot{q},\Phi)\dot{q} + g(q,\Phi) = u, \label{dynamics}
\end{equation}
where $q\in\mathbb{R}^n$ is the vector of joint angles, $M(\cdot)\in\mathbb{R}^{n\times n}$, $C(\cdot)\in\mathbb{R}^{n\times n}$, and $g(\cdot)\in\mathbb{R}^{n}$ denote the mass matrix, Coriolis matrix, and the gravity term respectively, $u\in\mathbb{R}^{n}$ is the motor torque input, and $\Phi = [\phi_{1}^{T}, \cdots, \phi_{n}^{T}]^{T}\in\mathbb{R}^{10n}$ is the vector of inertial parameters of each links, the technical goal of adaptive control is to design a control input $u$ that ensures global convergence of trajectory tracking error even in the presence of model uncertainty. In this paper, we confine our interest to the class of adaptive controllers that assumes only the model uncertainty in inertial parameter $\Phi$. Furthermore, as the term ``adaptive"  indicates, smooth time-varying estimates of the true model parameters are respected together with the trajectory tracking controller, which differentiates from class of discontinuous robust controllers using fixed parameter estimates with known uncertainty bound usually leading to chattering problem.
\subsection{Adaptive Computed Torque Control[John J.Crag et al]}
The control input for Adaptive Computed Torque Control, or so called Adaptive Inverse Dynamics, is given by,
\begin{align*}
u =& M(q,\hat{\Phi})\{\ddot{q}_{d}-K_{v}(\dot{q}-\dot{q}_{d})-K_{p}(q-q_{d})\} \nonumber\\
&+C(q,\dot{q},\hat{\Phi})\dot{q}+g(q,\hat{\Phi}),
\end{align*}
where $K_p\in\mathbb{R}^{n\times n}$ and $K_{v}\in\mathbb{R}^{n\times n}$ are diagonal matrices of positive gains, $q_{d}$ is the given reference trajectory and $\hat{\Phi}$ is the estimate of the true inertial parameter $\Phi$, whose update law would be clarified later. Then using the linear property of inertial parameters in dynamic equations, the closed loop dynamics becomes
\begin{equation}
\ddot{\tilde{q}}+K_{v}\dot{\tilde{q}}+K_{p}\tilde{q}=M(q,\hat{\Phi})^{-1}Y(q,\dot{q},\ddot{q})\tilde{\Phi} \label{ACTC_CL_dyn}
\end{equation}
, where $Y$ is the regressor function and $\tilde{\Phi} =\hat{\Phi}-\Phi$. The state space formulation of (\ref{ACTC_CL_dyn}) with augmented state vector defined as $e=[\tilde{q}^{T}, \dot{\tilde{q}}^{T}]^{T}$ is given by 
\begin{equation*}
\dot{e} = Ae+BM(q,\hat{\Phi})^{-1}Y(q,\dot{q},\ddot{q})\tilde{\Phi},
\end{equation*}
where $A = \left[\begin{array}{cc} 0 & I \\ -K_{p} & -K_{v}\end{array}\right]$ is a Hurwitz matrix and $B=\left[\begin{array}{c} 0 \\ I\end{array}\right]$. Then we may choose some $Q \in \mathcal{P}(n)$ and let $P \in \mathcal{P}(n)$ be the unique symmetric positive definite matrix satisfying the Lyapunov equation
\begin{equation*}
A^{T}P + PA = -Q.
\end{equation*}
Now we define the Lyapunov function candidate as the following:
\begin{equation*}
V = e^{T}Pe + \tilde{\Phi}^{T}\Gamma\tilde{\Phi},
\end{equation*}
where $\Gamma$ is a constant symmetric positive definite matrix.
Then $\dot{V}$ is calculated as,
\begin{equation*}
\dot{V} = -e^{T}Qe + 2\tilde{\Phi}^{T}\{Y(q,\dot{q},\ddot{q})^{T}M(q,\hat{\Phi})^{-1}B^{T}Pe+\Gamma\dot{\hat{\Phi}}\},
\end{equation*}
using the fact that $\dot{\tilde{\Phi}} = \dot{\hat{\Phi}}$, since $\Phi$ is constant. Choosing the parameter update law as,
\begin{equation}
\dot{\hat{\Phi}} = -\Gamma^{-1}Y(q,\dot{q},\ddot{q})^{T}M(q,\hat{\Phi})^{-1}B^{T}Pe, \label{ACTC_parameter_update}
\end{equation}
we have 
\begin{equation*}
\dot{V} = -e^{T}Qe \leq 0 .
\end{equation*}
From the Lyapunov stability analysis it follows that position tracking error $e$ converges to zero asymptotically and the parameter estimate error $\tilde{\Phi}^{T}\Gamma\tilde{\Phi}$ remains bounded. Noth that measurement of joint accelaration $\ddot{q}$ and invertibility of estimated mass matrix $M(q,\hat{\Phi})$ are required to implement the adaptation law (\ref{ACTC_parameter_update}), which is both practically hard to achieve. The Passivity-based Adaptive Control method proposed by Slotine and Li [] can remove both of these impediments.
% However, the invertibility of estimated mass matrix is related to the physical consistency of inertial parameters and note that the Passivity-based Adaptive Control still do not guarantee the non-singularity or positive definiteness of $M(q,\hat{\Phi})$.
% \subsection{(Robust Adaptive Computed Torque Control)}
\subsection{Passivity-based Adaptive Control[Slotine and Li]}
The control input for Passivity-based Adaptive Control is given by,
\begin{equation*}
u = M(q,\hat{\Phi})a + C(q,\dot{q},\hat{\Phi})v + g(q,\hat{\Phi})-Kr,
\end{equation*}
where the vectors $v,a,r\in\mathbb{R}^{n}$ are given as
\begin{align*}
v &= \dot{q}_{d}-\Lambda\tilde{q}\\
a &= \dot{v} = \ddot{q}_{d}-\Lambda\dot{\tilde{q}}\\
r &= \dot{q}-v = \dot{\tilde{q}}+\Lambda\tilde{q}
\end{align*}
and $K$ and $\Lambda$ are diagonal matrices of constant positive gains. Then the closed loop dynamics is given by,
\begin{equation*}
M(q,\Phi)\dot{r} + C(q,\dot{q},\Phi)r+Kr = Y(q,\dot{q},a,v)\tilde{\Phi}.
\end{equation*}
Here we introduce the Lyapunov function candidate as follows:
\begin{equation}
V = \frac{1}{2}r^{T}M(q,\Phi)r + \tilde{q}^{T}\Lambda K\tilde{q}+\frac{1}{2}\tilde{\Phi}^{T}\Gamma\tilde{\Phi} \label{Slotine_lyapunov}
\end{equation}
where $\Gamma$ is again set to be a constant symmetric positive definite matrix.
Using the skew-symmetry property of the matrix $\dot{M} - 2C$ [], $\dot{V}$ is calculated as,
\begin{equation*}
\dot{V} = -\tilde{q}^{T}\Lambda K \Lambda\tilde{q} - \dot{\tilde{q}}^{T}K\dot{\tilde{q}} + \tilde{\Phi}^{T}\{\Gamma\dot{\hat{\Phi}} + Y^{T}r\}.
\end{equation*}
Choosing the parameter update law as,
\begin{equation}
\dot{\hat{\Phi}} = -\Gamma^{-1}Y(q,\dot{q},a,v)^{T}r, \label{PBAC_parameter_update}
\end{equation}
we have
\begin{equation*}
\dot{V} = -\left[\begin{array}{c} \tilde{q} \\ \dot{\tilde{q}}\end{array}\right]^{T}\left[\begin{array}{cc} \Lambda K \Lambda & 0 \\ 0 & K \end{array}\right]\left[\begin{array}{c} \tilde{q} \\ \dot{\tilde{q}}\end{array}\right] = -e^{T}Qe \leq 0.
\end{equation*}
It can also be shown that position tracking error $e$ converges to zero asymptotically and parameter estimate error remains bounded. Note that the acceleration measurements nor the inverse matrix $M(q,\hat{\Phi})$ is present in the adaptation law.

\section{Contribution}
Lyapunov stability analysis provides a technical way of assessing the stability of the closed loop system, where in the investigation of particular choice of Lyapunov function candidate is demanded. However, a natural choice of Lyapunov function candidate at firsthand can sometimes adversely help in designing the natural stabilizing control law for the open loop system. Especially for a large class of mechanical systems, a branch of passivity-based control methods that exploits the skew-symmetry of $\dot{M} - 2C$ have been developed by considering the choice of energy-like Lyapunov function candidate, i.e. system's kinetic, elastic energy related functions as also in (\ref{Slotine_lyapunov}). Such method allows the closed loop system to inherit the intrinsic physical properties of the original system, rather than entirely substituting the original dynamics with that of virtual spring- damper system through feedback linearization as in computed-torque control. Slotine and Li, as described in the former section, has shown that such physically motivating control design can also be successfully applied to adaptive manipulator control.

Meanwhile, general class of globally convergent adaptive control for robot manipulators discussed in the former section consider the Lyapunov function candidate $V$ as the summation of tracking error term $V_t$ and parameter error term $V_p$:
\begin{equation}
V = \underbrace{V_t(q,\dot{q},q_{d},\dot{q_{d}},\Phi)}_{\text{(tracking error)}} + \underbrace{V_p(\Phi, \hat{\Phi})}_{\text{(parameter error)}}.
\end{equation}
Here, as opposed to the choice of tracking error term $V_{t}$, the natural choice of parameter error term $V_{p}$ is often overlooked, in most cases naively considering quadratic parameter error of the from $\tilde{\Phi}^{T}\Gamma\tilde{\Phi}$ with constant positive definite matrix $\Gamma$. The fallacy of measuring distance on inertial parameters in a non-coordinate-invariant way as $\tilde{\Phi}^{T}\Gamma\tilde{\Phi}$ has been discussed in [], highly susceptible to a fatal scale problem and thereby leading to a physically inconsistent estimation of the parameters. Such problem can also be pervasive in adaptive control. Indeed, for the case of adaptive computed-torque control where the invertibility of mass matrix $M(q,\hat{\Phi})$ is demanded, several projection based modification of the adaptation law has been proposed to keep estimated inertial parameters to be physically consistent, which then sufficiently guarantees the invertibility of $M(q,\hat{\Phi})$. However, such method is somehow an ad-hoc remedy to the problem and can produce non-smooth torque input. On the other hand, Slotine and Li have suggested composite version of their passivity based method, where they take into account additional filtered torque prediction error by appropriately updating the matrix $\Gamma$, leading to more well-conditioned value of $\Gamma$ over the time.

However, to the best of our knowledge, none of the adaptation laws so far respect the physical consistency of the estimated parameters in an intrinsic manner, while also producing the smooth torque input. In the following section, we argue that these desirable aspects can be achieved by simply considering the natural and coordinate invariant choice of smooth parameter error term $V_{p}$ at firsthand, resulting in natural gradient-like adaptation law of the parameters.
\subsection{Bregman Divergence as a Lyapunov Function Candidate}
A seemingly possible and natural choice of $V_{p}$ could be a geodesic distance metric as defined in [], i.e. 
\begin{equation}
V_{p}(\Phi,\hat{\Phi}) = \sum_{i=1}^{n} d_{\mathcal{M}}(\phi_{i}, \hat{\phi_{i}}). \label{geodesic_distance}
\end{equation}
However, it turns out that such choice fails to argebraically enforce the negative-definiteness of $\dot{V}$. Recall that the time derivative of tracking error related Lyapunov function candidate $V_{t}$ is derived in the following form:
\begin{align}
\dot{V_t} &= -e^{T}Qe +\tilde{\Phi}^{T}b \label{dot_V_t}
% &= -e^{T}Qe + \tilde{\Phi}^{T}b + \tilde{\Phi}^{T}g(\dot{\hat{\Phi}}, \hat{\Phi})\nonumber\\
% &= -e^{T}Qe + \tilde{\Phi}^{T}\{b+ g(\dot{\hat{\Phi}}, \hat{\Phi})\nonumber\}
\end{align}
, where for the Adaptive Computed Torque Control $b = Y^T\hat{M}^{-1}B^{T}Pe$, and for the Passivity-based Adaptive Control $b = Y^{T}r$. In order to make the time derivative of the entire Lyapunov function candidate $V = V_{t}+V_{p}$ to be negative definite, time derivative of $V_{p}$ is algebraically required to be of the form,
\begin{equation}
\dot{V_{p}}(\Phi, \hat{\Phi})=\tilde{\Phi}^{T}g(\dot{\hat{\Phi}}, \hat{\Phi}) \label{dot_V_p}
\end{equation}
, resulting in the adaptation law $g(\dot{\hat{\Phi}}, \hat{\Phi}) = - b$ and eventually $\dot{V} =\dot{V_{t}}+\dot{V_{p}}= -e^{T}Qe \leq 0$. When $V_{p}$ is set as convention, i.e. $V_{p} = \tilde{\Phi}^{T}\Gamma\tilde{\Phi}$, then $g(\dot{\hat{\Phi}}, \hat{\Phi}) = \Gamma \dot{\hat{\Phi}}$. Note here that $g$ should be a function of only estimated parameters $\hat{\Phi}$ and their time derivatives $\dot{\hat{\Phi}}$ and not include true parameter $\Phi$, since it is not known $a \ priori$. With the choice of $V_{p}$ in (\ref{geodesic_distance}), its time derivative fails to be expressed in the form as (\ref{dot_V_p}) at all for its nonlinearity.

We now claim that the Bregman divergence based distance measure discussed in (\ref{Bregman_div}) allows a valid Lyapunov function candidate of $V_{p}$ whose time derivative can actually be derived in the form (\ref{dot_V_p}).
\begin{proposition}
A function $V_{p}$ defined as
\begin{equation}
V_{p}(\Phi,\hat{\Phi}) = \gamma \sum_{i=1}^n D_{F(\mathcal{P}(4))}(f(\phi_{i}) || f(\hat{\phi_{i}})) \label{bregman_V_p}
%\left[\log\bigg(\frac{|f(\hat{\phi_{i}})|}{|f(\phi_{i})|}\bigg) + \mathrm{tr}(f(\hat{\phi_{i}})^{-1}f(\phi_{i})) - 4\right] 
\end{equation}
 for constant positive scalar gain $\gamma$ and true parameter vector $\Phi$ is a valid Lyapunov function candidate of $\hat{\Phi}$, which is differentiated in the form (\ref{dot_V_p}).

\begin{proof}
Firstly, it can be observed that $V_{p}$ is a symmetric function of a set of eigenvalues $\lambda_{i}^{j}$ ($j=1,2,3,4$) of matrices $f(\hat{\phi_{i}})^{-1}f(\phi_{i})$, which implies coordinate invariance of $V_{p}$. Explicitly $V_{p}$ is given by
\begin{equation*}
V_{p} = \gamma\sum_{i=1}^{n}\sum_{j=1}^{4}[ - \log(\lambda_{i}^{j}) + \lambda_{i}^{j} - 1].
\end{equation*}
Since a scalar function $-\log(x)+x-1$ is always non-negative, $V_{p}\geq 0$ is a valid Lyapunov function candidate, having zero only when $\lambda_{i}^{j} = 1$ for all $i,j$, i.e. $ \hat{\Phi} = \Phi$.

The time derivative of $V_{p}$ is given by,
\begin{align}
\dot{V_{p}} =&\gamma\frac{d}{dt}\sum_{i=1}^n\left[\log\bigg(\frac{|f(\hat{\phi_{i}})|}{|f(\phi_{i})|}\bigg) + \mathrm{tr}(f(\hat{\phi_{i}})^{-1}f(\phi_{i})) - 4\right] \nonumber\\
=&\gamma\sum_{i=1}^n \mathrm{tr}\bigg(f(\hat{\phi_{i}})^{-1}f(\dot{\hat{\phi_{i}}})-f(\hat{\phi_{i}})^{-1}f(\dot{\hat{\phi_{i}}})f(\hat{\phi_{i}})^{-1}f(\phi_{i})\bigg)\nonumber\\
=& \gamma\sum_{i=1}^n \mathrm{tr}\bigg(f(\hat{\phi_{i}})^{-1}f(\dot{\hat{\phi_{i}}})f(\hat{\phi_{i}})^{-1}(f(\hat{\phi_{i}})-f(\phi_{i}))\bigg)\nonumber\\
=&\gamma\sum_{i=1}^n \mathrm{tr}\bigg(\big[f(\hat{\phi_{i}})^{-1}f(\dot{\hat{\phi_{i}}})f(\hat{\phi_{i}})^{-1}\big]f(\hat{\phi_{i}}-\phi_{i})\bigg)\label{dot_V_p_bregman}
\end{align}
, where we used the fact that $\frac{d}{dt}f(\hat{\phi})=f(\dot{\hat{\phi}})$ and $f(\hat{\phi})-f(\phi) = f(\hat{\phi}-\phi)$ which hold by the linearity of mapping $f$. Again by the linearity of $f$, $\dot{V_{p}}$ calculated above can be expressed in the form (\ref{dot_V_p}).
\end{proof}
\end{proposition}
\subsection{Natural Adaptation Law}
Based on the choice of $V_{p}$ as in (\ref{bregman_V_p}), we now propose a novel adaptation law applicable to any adaptive control methods in which the time derivative of tracking error related Lyapunov function candidate $V_{t}$ is expressible in the form (\ref{dot_V_t}).
\begin{proposition}
Given the control law that results in the time derivative of $V_{t}$ of the form (\ref{dot_V_t}), the following adaptation law:
\begin{equation}
\dot{\hat{P_i}} = -\frac{1}{\gamma}\hat{P_i} B_i \hat{P_i}, \qquad i = 1, \cdots, n \label{natural_adaptation_1}
\end{equation}
or equivalently,
\begin{equation}
\frac{d}{dt}(\hat{P_i}^{-1}) = \frac{1}{\gamma}B_i, \qquad i = 1, \cdots, n \label{natural_adaptation_2}
\end{equation}
, where $\hat{P_i} = f(\hat{\phi_i})$ and $B_i$ are unique symmetric matrices that satisfies the relation $\tilde{\Phi}^{T}b = \sum_{i=1}^n \mathrm{tr}(f(\tilde{\phi_i})B_i)$, gurantees the asymptotic convergence of tracking error to zero, bounded parameter error and also physical consistency of the estimated parameter $\hat{\Phi}$ over the time given that the initial estimate is chosen to be physically consistent.

\begin{proof}
Consider the valid Lyapunov function candidate,
\begin{equation*}
V = V_t + V_p
\end{equation*}
, where $V_p$ is defined by (\ref{bregman_V_p}). Then,
\begin{align*}
\dot{V} &= \dot{V_t} + \dot{V_p}\\
 &= -e^{T}Qe + \tilde{\Phi}^{T}b + \gamma\sum_{i=1}^n \mathrm{tr}\big(\hat{P_i}^{-1}\dot{\hat{P_i}}\hat{P_i}^{-1}\tilde{P_i}\big)\\
 &=-e^{T}Qe + \sum_{i=1}^n\mathrm{tr}(\tilde{P_i}B_i) + \gamma\sum_{i=1}^n \mathrm{tr}\big(\hat{P_i}^{-1}\dot{\hat{P_i}}\hat{P_i}^{-1}\tilde{P_i}\big)\\
 &= -e^{T}Qe + \sum_{i=1}^n \mathrm{tr}\big([B_i +\gamma\hat{P_i}^{-1}\dot{\hat{P_i}}\hat{P_i}^{-1}]\tilde{P_i}\big)
\end{align*}
holds from (\ref{dot_V_t}), (\ref{dot_V_p_bregman}) and $\tilde{P_i} = f(\tilde{\phi_i}) = \hat{P_i}-P_i$. From the adaptation rule defined by (\ref{natural_adaptation_1}) or equivalently (\ref{natural_adaptation_2}),
\begin{equation*}
\dot{V} = -e^{T}Qe \leq 0
\end{equation*}
holds and the asymptotic convergence of tracking error can be shown in the same manner as before. Moreover, since $\dot{V} \leq 0$, $V$ is bounded and therefore $V_p = V - V_t \leq V$ is always bounded over the time. Meanwhile, $V_p(0)$ is truly bounded, since initial parameter estimate is set to be physically consistent, i.e. $\hat{P_i}(0) \in \mathcal{P}(4)$. If there exists some time instance $T>0$ where $\hat{P_i}(T)$ is not positive definite, then by the continuity there exists time instance $0<t_0\leq T$ such that $\hat{P_i}(t_0)$ is on the boundary of $\mathcal{P}(4)$, i.e. at least one of the eigenvalues of $\hat{P_i}(t_0)$ be zero. Then this contradicts the fact that $V_p(t)$ be bounded for all $t>0$, since $V_p(t_0)$ is infinity. Therefore, the present adaptation law always gurantees the physical consistency of $\hat{\Phi} \sim \{\hat{P_i}\}_{i=1}^n$.
\end{proof}
\end{proposition}
\begin{remark}
As was shown, the adaptation law (\ref{natural_adaptation_1}) or (\ref{natural_adaptation_2}) theoretically assures the physical consistency of the estimated parameters, that is $\hat{P_i}$ are always updated to be positive definite regardless of $\gamma$ and $B_{i}$. However, from the view of practical implementatin of the adaptation law, when preset gain $1/\gamma$ or $B_{i}$ in some time instance is large relative to the discrete time step size, the naive Euler integration on Euclidean space as,
\begin{equation*}
\hat{P_i}(t+\Delta t) = \hat{P_i}(t) - \frac{1}{\gamma}\hat{P_i}(t)B_{i}(t)\hat{P_i}(t)\cdot\Delta t
\end{equation*}
or,
\begin{equation*}
\hat{P_i}(t+\Delta t) = (\hat{P_i}(t)^{-1} + \frac{1}{\gamma}B_{i}(t)\cdot\Delta t)^{-1}
\end{equation*}
, might not gurantee the positive definiteness of $\hat{P}_i$ over the time, although the continuous version of the algorithm guarantees such aspect. Therefore, we recommend to update $\hat{P}_i$ by exponential map or geodesic path on $\mathcal{M} \simeq \mathcal{P}(4)$ as,
\begin{align}
\hat{P_i}(t+\Delta t) &= \mathrm{Exp}_{\hat{P_i}(t)}\bigg(-\frac{1}{\gamma}\hat{P_i}(t)B_{i}(t)\hat{P_i}(t)\cdot\Delta t\bigg) \nonumber\\
&=\hat{P_i}(t)^{1/2}e^{-\frac{\Delta t}{\gamma}\hat{P_i}(t)^{1/2}B_i(t)\hat{P_i}(t)^{1/2}}\hat{P_i}(t)^{1/2} \label{natural_adaptation_discrete}
\end{align}
, which always gurantees the positive definiteness of $\hat{P}_i$. One might try to update (\ref{ACTC_parameter_update}) or (\ref{PBAC_parameter_update}) also with exponential map on the chance of guranteeing the physical consistency of estimated parameters. However, this would rather deter the stability of the system behaviour, since the update rule in (\ref{ACTC_parameter_update}) or (\ref{PBAC_parameter_update}) usually do require estimation of parameters to be physically inconsistent during the adaptation.
\end{remark}
\begin{remark}
Natural gradient $\nabla^2 F = G$, $[\nabla^2 F]^{-1}\frac{\partial f}{\partial x}$
\end{remark}
\begin{remark}
Passivity preserving integrator of $\tilde{\Phi}$ to $b$ [Applied Nonlinear Control, page. 421]
\end{remark}
\begin{remark}
Extension to composite adaptive control
\end{remark}
%\section{Recursive Dynamics for torque input}
\section{Experiment Results}
\section{Introduction}
Many authors submitting to research journals use \LaTeXe\ to
prepare their papers. This paper describes the
\textsf{\journalclass} class file which can be used to convert
articles produced with other \LaTeXe\ class files into the correct
form for submission to \textit{SAGE Publications}.

The \textsf{\journalclass} class file preserves much of the
standard \LaTeXe\ interface so that any document which was
produced using the standard \LaTeXe\ \textsf{article} style can
easily be converted to work with the \textsf{\journalclassshort}
style. However, the width of text and typesize will vary from that
of \textsf{article.cls}; therefore, \textit{line breaks will change}
and it is likely that displayed mathematics and tabular material
will need re-setting.

In the following sections we describe how to lay out your code to
use \textsf{\journalclass} to reproduce much of the typographical look of
the \textit{SAGE} journal that you wish to submit to. However, this paper is not a guide to
using \LaTeXe\ and we would refer you to any of the many books
available (see, for example, \cite{R1}, \cite{R2} and \cite{R3}).

\section{The three golden rules}
Before we proceed, we would like to stress \textit{three golden
rules} that need to be followed to enable the most efficient use
of your code at the typesetting stage:
\begin{enumerate}
\item[(i)] keep your own macros to an absolute minimum;

\item[(ii)] as \TeX\ is designed to make sensible spacing
decisions by itself, do \textit{not} use explicit horizontal or
vertical spacing commands, except in a few accepted (mostly
mathematical) situations, such as \verb"\," before a
differential~d, or \verb"\quad" to separate an equation from its
qualifier;

\item[(iii)] follow the journal reference style.
\end{enumerate}

\section{Getting started} The \textsf{\journalclassshort} class file should run
on any standard \LaTeXe\ installation. If any of the fonts, style
files or packages it requires are missing from your installation,
they can be found on the \textit{\TeX\ Collection} DVDs or downloaded from
CTAN.

\begin{figure*}
\setlength{\fboxsep}{0pt}%
\setlength{\fboxrule}{0pt}%
\begin{center}
\begin{boxedverbatim}
\documentclass[<options>]{sagej}

\begin{document}

\runninghead{<Author surnames>}

\title{<Initial capital only>}

\author{<An Author\affilnum{1},
Someone Else\affilnum{2} and
Perhaps Another\affilnum{1}>}

\affiliation{<\affilnum{1}First and third authors' affiliation\\
\affilnum{2}Second author affiliation>}

\corrauth{<Corresponding author's name and full postal address>}

\email{<Corresponding author's email address>}

\begin{abstract}
<Text>
\end{abstract}

\keywords{<List keywords>}

\maketitle

\section{Introduction}
.
.
.
\end{boxedverbatim}
\end{center}
\caption{Example header text.\label{F1}}
\end{figure*}

\section{The article header information}
The heading for any file using \textsf{\journalclass} is shown in
Figure~\ref{F1}. You must select options for the trim/text area and
the reference style of the journal you are submitting to.
The choice of \verb+options+ are listed in Table~\ref{T1}.

\begin{table}[h]
\small\sf\centering
\caption{The choice of options.\label{T1}}
\begin{tabular}{lll}
\toprule
Option&Trim and font size&Columns\\
\midrule
\texttt{shortAfour}& 210 $\times$ 280 mm, 10pt& Double column\\
\texttt{Afour} &210 $\times$ 297 mm, 10pt& Double column\\
\texttt{MCfour} &189 $\times$ 246 mm, 10pt& Double column\\
\texttt{PCfour} &170 $\times$ 242 mm, 10pt& Double column\\
\texttt{Royal} &156 $\times$ 234 mm, 10pt& Single column\\
\texttt{Crown} &7.25 $\times$ 9.5 in, 10pt&Single column\\
\texttt{Review} & 156 $\times$ 234 mm, 12pt & Single column\\
\bottomrule
\end{tabular}\\[10pt]
\begin{tabular}{ll}
\toprule
Option&Reference style\\
\midrule
\texttt{sageh}&SAGE Harvard style (author-year)\\
\texttt{sagev}&SAGE Vancouver style (superscript numbers)\\
\texttt{sageapa}&APA style (author-year)\\
\bottomrule
\end{tabular}
\end{table}

For example, if your journal is short A4 sized, uses Times fonts and has Harvard style references then you would need\\
{\small\verb+\documentclass[ShortAfour,times,sageh]{sagej}+}

Most \textit{SAGE} journals are published using Times fonts but if for any reason you have a problem using Times you can
easily resort to Computer Modern fonts by removing the
\verb"times" option.

\subsection{`Review' option}
Some journals (for example, \emph{Journal of the Society for Clinical Trials}) require that 
papers are set single column and with a larger font size to help with the review process. 
If this is a requirement for the journal that you are submitting to, just add the \verb+Review+ option to the \verb+\documenclass[]{sagej}+ line.

\subsection{Remarks}
\begin{enumerate}
\item[(i)] In \verb"\runninghead" use `\textit{et~al.}' if there
are three or more authors.

\item[(ii)] For multiple author papers please note the use of \verb"\affilnum" to
link names and affiliations. The corresponding author details need to be included using the
\verb+\corrauth+ and \verb+\email+ commands.

\item[(iii)] For submitting a double-spaced manuscript, add
\verb"doublespace" as an option to the documentclass line.

\item[(iv)] The abstract should be capable of standing by itself,
in the absence of the body of the article and of the bibliography.
Therefore, it must not contain any reference citations.

\item[(v)] Keywords are separated by commas.

\item[(vi)] If you are submitting to a \textit{SAGE} journal that requires numbered sections (for example, IJRR), please add the command
  \verb+\setcounter{secnumdepth}{3}+ just above the \verb+\begin{document}+ line.

\end{enumerate}


\section{The body of the article}

\subsection{Mathematics} \textsf{\journalclass} makes the full
functionality of \AmS\/\TeX\ available. We encourage the use of
the \verb"align", \verb"gather" and \verb"multline" environments
for displayed mathematics. \textsf{amsthm} is used for setting
theorem-like and proof environments. The usual \verb"\newtheorem"
command needs to be used to set up the environments for your
particular document.

\subsection{Figures and tables} \textsf{\journalclass} includes the
\textsf{graphicx} package for handling figures.

Figures are called in as follows:
\begin{verbatim}
\begin{figure}
\centering
\includegraphics{<figure name>}
\caption{<Figure caption>}
\end{figure}
\end{verbatim}

For further details on how to size figures, etc., with the
\textsf{graphicx} package see, for example, \cite{R1}
or \cite{R3}.

The standard coding for a table is shown in Figure~\ref{F2}.

\begin{figure}
\setlength{\fboxsep}{0pt}%
\setlength{\fboxrule}{0pt}%
\begin{center}
\begin{boxedverbatim}
\begin{table}
\small\sf\centering
\caption{<Table caption.>}
\begin{tabular}{<table alignment>}
\toprule
<column headings>\\
\midrule
<table entries
(separated by & as usual)>\\
<table entries>\\
.
.
.\\
\bottomrule
\end{tabular}
\end{table}
\end{boxedverbatim}
\end{center}
\caption{Example table layout.\label{F2}}
\end{figure}

\subsection{Cross-referencing}
The use of the \LaTeX\ cross-reference system
for figures, tables, equations, etc., is encouraged
(using \verb"\ref{<name>}" and \verb"\label{<name>}").

\subsection{End of paper special sections}
Depending on the requirements of the journal that you are submitting to,
there are macros defined to typeset various special sections. 

The commands available are:
\begin{verbatim}
\begin{acks}
To typeset an
  "Acknowledgements" section.
\end{acks}
\end{verbatim}

\begin{verbatim}
\begin{biog}
To typeset an
  "Author biography" section.
\end{biog}
\end{verbatim}

\begin{verbatim}
\begin{biogs}
To typeset an
  "Author Biographies" section.
\end{biogs}
\end{verbatim}

%\newpage

\begin{verbatim}
\begin{dci}
To typeset a "Declaration of
  conflicting interests" section.
\end{dci}
\end{verbatim}

\begin{verbatim}
\begin{funding}
To typeset a "Funding" section.
\end{funding}
\end{verbatim}

\begin{verbatim}
\begin{sm}
To typeset a
  "Supplemental material" section.
\end{sm}
\end{verbatim}

\subsection{Endnotes}
Most \textit{SAGE} journals use endnotes rather than footnotes, so any notes should be coded as \verb+\endnote{<Text>}+.
Place the command \verb+\theendnotes+ just above the Reference section to typeset the endnotes.

To avoid any confusion for papers that use Vancouver style references,  footnotes/endnotes should be edited into the text.

\subsection{References}
Please note that the files \textsf{SageH.bst} and \textsf{SageV.bst} are included with the class file
for those authors using \BibTeX. 
The files work in a completely standard way, and you just need to uncomment one of the lines in the below example depending on what style you require:
\begin{verbatim}
%%Harvard (name/date)
%\bibliographystyle{SageH}
%%Vancouver (numbered)
%\bibliographystyle{SageV} 
\bibliography{<YourBibfile.bib>} 
\end{verbatim}

%\section{Support for \textsf{\journalclass}}
%We offer on-line support to participating authors. Please contact
%us via e-mail at \dots
%
%We would welcome any feedback, positive or otherwise, on your
%experiences of using \textsf{\journalclass}.

\section{Copyright statement}
Please  be  aware that the use of  this \LaTeXe\ class file is
governed by the following conditions.

\subsection{Copyright}
Copyright \copyright\ \volumeyear\ SAGE Publications Ltd,
1 Oliver's Yard, 55 City Road, London, EC1Y~1SP, UK. All
rights reserved.

\subsection{Rules of use}
This class file is made available for use by authors who wish to
prepare an article for publication in a \textit{SAGE Publications} journal.
The user may not exploit any
part of the class file commercially.

This class file is provided on an \textit{as is}  basis, without
warranties of any kind, either express or implied, including but
not limited to warranties of title, or implied  warranties of
merchantablility or fitness for a particular purpose. There will
be no duty on the author[s] of the software or SAGE Publications Ltd
to correct any errors or defects in the software. Any
statutory  rights you may have remain unaffected by your
acceptance of these rules of use.

\begin{acks}
This class file was developed by Sunrise Setting Ltd,
Brixham, Devon, UK.\\
Website: \url{http://www.sunrise-setting.co.uk}
\end{acks}

\begin{thebibliography}{99}
\bibitem[Kopka and Daly(2003)]{R1}
Kopka~H and Daly~PW (2003) \textit{A Guide to \LaTeX}, 4th~edn.
Addison-Wesley.

\bibitem[Lamport(1994)]{R2}
Lamport~L (1994) \textit{\LaTeX: a Document Preparation System},
2nd~edn. Addison-Wesley.

\bibitem[Mittelbach and Goossens(2004)]{R3}
Mittelbach~F and Goossens~M (2004) \textit{The \LaTeX\ Companion},
2nd~edn. Addison-Wesley.

\end{thebibliography}

\end{document}
